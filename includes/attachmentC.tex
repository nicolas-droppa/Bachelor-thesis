\begin{trivlist}
  \item{\textbf{Abstrakt}} -- Krátke zhrnutie hlavných bodov dokumentu, vrátane cieľov, metodológie a~výsledkov.

  \item{\textbf{Bod (pt)}} -- Jednotka merania veľkosti písma; 1 bod sa rovná približne 0,352\,mm.
  
  \item{\textbf{Bold}} -- Písmo, ktoré je tučné, používané na zvýraznenie dôležitých častí textu a~v~nadpisoch.
  
  \item{\textbf{Citácia}} -- Uvedenie zdroja informácií v texte, ktoré umožňuje identifikáciu pôvodného materiálu.
  
  \item{\textbf{Font}} -- Typ písma, ktorý sa používa na zobrazenie textu. Rôzne fonty majú odlišný štýl a~vzhľad.
  
  \item{\textbf{Graf}} -- Vizualizácia údajov, ktorá zobrazuje vzťah medzi premennými.
  
  \item{\textbf{Kapitola}} -- Hlavná časť dokumentu, ktorá sa zaoberá konkrétnou témou a~je zvyčajne označená číslom alebo názvom.
  
  \item{\textbf{Korektúra}} -- Proces revízie a~opravy textu s~cieľom odstrániť chyby.
  
  \item{\textbf{Kurzíva}} -- Typ písma, ktorý je sklonený doprava a~používa sa na zvýraznenie textu, názvov alebo cudzích výrazov.
  
  \item{\textbf{Metodológia}} -- Opis metód a~techník použitých pri výskume alebo analýze.
  
  \item{\textbf{Normostrana}} -- Základná jednotka pre meranie dĺžky textu, ktorá zvyčajne obsahuje 1\,800 znakov vrátane medzier.
  
  \item{\textbf{Obrázok}} -- Grafický prvok alebo fotografia, ktorá ilustruje text.
  
  \item{\textbf{Odsadenie}} -- Vzdialenosť prvého riadku odseku od okraja stránky.
  
  \item{\textbf{Odsek}} -- Základná jednotka textu, ktorá obsahuje súvislé myšlienky. V~\LaTeX-u sa odseky oddeľujú prázdnym riadkom v~zdrojovom texte.
  
  \item{\textbf{Plagiátorstvo}} -- Prezentovanie cudzej práce alebo myšlienok ako vlastných bez riadneho citovania.
  
  \item{\textbf{Rez písma}} -- Variant typu písma, ktorý určuje štýlové úpravy, ako sú normálny, tučný (bold), kurzíva (italic) alebo podčiarknutie.
  
  \item{\textbf{Riadkovanie}} -- Vzdialenosť medzi jednotlivými riadkami textu.
  
  \item{\textbf{Sadzba}} -- Proces usporiadania textu a~grafiky na stránke.
  
  \item{\textbf{Súbor}} -- Elektronický dokument, ktorý obsahuje text alebo dáta.
  
  \item{\textbf{Šablóna}} -- Prednastavené formátovanie pre rôzne typy textu, ktoré zabezpečuje jednotný vzhľad.
  
  \item{\textbf{Tabuľka}} -- Usporiadanie údajov do riadkov a~stĺpcov.
  
  \item{\textbf{Titulný list}} -- Prvá stránka dokumentu, ktorá obsahuje základné informácie, ako je názov práce a~meno autora.
  
  \item{\textbf{Typografia}} -- Umenie a~technika usporiadania textu tak, aby bol esteticky príťažlivý a~čitateľný.
  
  \item{\textbf{Úvod}} -- Prvá časť dokumentu, ktorá predstavuje tému a~ciele výskumu.
  
  \item{\textbf{Vedecká práca}} -- Akademický dokument, ktorý prezentuje výsledky výskumu, analýzy alebo diskusie o~špecifickej téme.
  
  \item{\textbf{Zdroje primárne a sekundárne}} -- Primárne zdroje sú originálne dokumenty; sekundárne zdroje interpretujú alebo analyzujú primárne zdroje.
\end{trivlist}